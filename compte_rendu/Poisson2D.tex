\documentclass{article}

\usepackage[latin1]{inputenc}
\usepackage[T1]{fontenc}
\usepackage[french]{babel}
\usepackage{graphicx}
\usepackage{geometry}
\usepackage{amsmath,amsfonts,amssymb}

\geometry{hmargin=2.5cm,vmargin=1.5cm}
\author{J.PERDIGON}

\begin{document}

\title{PPSC2020F - Projet I}
\maketitle

\newcommand{\xv}{\textbf{x}}


\section{Introduction}

On consid�re le probl�me de Poisson � 2 dimensions avec des conditions aux bords de Dirichlet homog�nes
\begin{equation}
  \begin{split}
  -\Delta u(\xv) = f(\xv) \ , \ \xv \in \Omega = (0,1) \times (0,1) \\
  u(\xv) = 0 \ , \ \xv \in \partial \Omega
  \end{split}
\end{equation}
On choisit d'approximer le probl�me � l'aide de diff�rences finies. L'espace est discr�tis� en une grille uniforme de ($n+1$) points dans chaque direction avec le pas de grille $h = 1/n$. L'op�rateur Laplacien $\Delta$ est quand � lui approxim� par:
\begin{equation}
  \Delta u(\xv) \approx \frac{u_{i+1,j} - 2u_{i,j} + u_{i-1,j}}{h^2} + \frac{u_{i,j+1} - 2u_{i,j} + u_{i,j-1}}{h^2}
\end{equation}
O� $u_{i,j} \approx  u(\xv_{i,j})$ repr�sente une approximation discrete de la solution, sur la grille d'espace d�finie pr�c�demment. $u_{i,j}$ peut �tre repr�sent� sous la forme d'une matrice $U$, de taille $(n-1) \times (n-1)$ et dont les indices $i$ et $j$ sont respectivement les indices de lignes et de colonnes (). 





\end{document}
